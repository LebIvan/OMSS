\newpage
\section{Билет 15. Совершенный газ. Адиабатический процесс. Полная система уравнений, описывающая движение нетеплопроводного идеального газа. Скорость звука. Число Маха. Критерий сжимаемости в стационарном случае. Квазиодномерные стационарные течения. Сопло Лаваля}
	

\textbf{Совершенным газом} называется газ, для которого выполнено:

$1) p = \rho R T$ - уравнение Клапейрона

$2) u = c_v T + const, c_v = const$

Здесь $R = \dfrac{R_0}{\mu}$, $R_0$ - постоянная, одинаковая для всех газов.

$\mu$ - молекулярный вес газа

$c_v$ - удельная теплоемкость в процессах с постоянным объемом частиц.

Если воспользоваться уравнением неразрывности (1), уравнением движения (2) и вторым законом термодинамики (3), 
x   то мы получим замкнутую систему уравнений для определения неизвестных функций $\rho, v, s$:
$$\dfrac{d \rho}{d t} + \rho \,div\, v = 0 $$
  $$  \rho \dfrac{d v}{d t} = \rho F - grad p $$
   $$ T \dfrac{d s}{d t} = q^{e}$$

Процесс называется \textbf{Адиабатическим}, если нет притока тепла ни к одной частице, то есть $q^{e} = 0$.

В случае, если процесс адиабатаический, то тогда $\dfrac{d s}{d t} = 0$, где $s$ - энтропия. Мы считаем, что $s = s(p,\rho)$, тогда можно сказать, что $p = p (s, \rho)$. Тогда $d p = \left(\dfrac{d p}{d s}\right)d s + \left(\dfrac{d p}{d \rho}\right) d \rho$. Производную $\left(\dfrac{d p}{d \rho}\right)$ обозначают $a^2$. Используя условие $\dfrac{d s}{d t} = 0$ получим, что $\dfrac{d p}{d t} = a^2 \dfrac{d \rho}{d t}$.

Если задана внутренняя энергия $u = u (\rho, s)$, то $p = \rho^2 \dfrac{d u}{d \rho}$ и $T d s = d u$. Тогда получаем : $d u = \dfrac{p}{\rho^2}d \rho + T d s$ - тождество Гиббса.

Подставим условия совершенного газа $ p = \rho R T$ и $u = c_v T + const, c_v = const$ и получим следующее выражение:
$$T d s = c_v d T - \dfrac{R T}{\rho}d \rho$$
Используя обозначение $\dfrac{R}{c_v} = \gamma - 1$ можно записать выражение для плотности энтропии совершенного газа: 
$$s = c_vln\dfrac{T}{\rho^{\gamma-1}} + C_1 = c_v ln\dfrac{p}{\rho^\gamma} + C$$

Полная система уравнений, описывающая движение нетеплопроводного идеального газа:
$$\dfrac{d \rho}{d t} + \rho \,div\, v = 0 $$
  $$  \rho \dfrac{d v}{d t} = \rho F - grad \,p $$
  $$c_v T\, ln\dfrac{p}{\rho^\gamma} = 0$$
  $$1) p = \rho R T$$

В случае адиабатического процесса $s = const$. Тогда $\dfrac{p}{\rho^\gamma} = const $. Значит $a^2 = \gamma \dfrac{p}{\rho}$ - скорость звука.

Наши допущения: газ идеальный, течение адиабатическое, стационарное и одномерное (квазистационарное и одномерное теченение). Пусть $S$ - сечение.
$$div\, \rho v = 0$$
$$\rho v S = const$$
Прдифференцировав последнее равенство получим $\dfrac{d \rho}{\rho} + \dfrac{d v}{v} + \dfrac{d S}{S} = 0$.
Вспомним, что $dp = a^2 d\rho$.
Уравнение движения Эйлера запишется в виде:$$v\, dv = -a^2 \dfrac{d \rho}{\rho}$$
В итоге получаем $(v^2 - a^2)\dfrac{d v}{v} = a^2 \dfrac{dS}{S}$. Следовательно $(M^2 - 1)\dfrac{d v}{v = \dfrac{d S}{S}}$.
\textbf{Число Маха} $M$ - это отношение величины скорости потока к скорости звука в рассматриваемой точке. $M = \dfrac{v}{a}$

Из уравнения ясно видна роль числа М как критерия сжимаемости: чем меньше М, тем меньше влияние изменения скорости  на относительное изменение объема.

1) Если M < 1 , знак dv противоположен знаку dS , то есть, при дозвуковом движении газа, так же, как для несжимаемой жидкости, с возрастанием площади сечения трубы скорость движения уменьшается, а при уменьшении сечения скорость увеличивается.

Если M > 1 , знак dv совпадает со знаком dS , то есть, при сверхзвуковом движении газа в сужающейся трубе движение замедляется, а в расширяющейся ускоряется. Поэтому для получения сверхзвуковой струи надо использовать специальный насадок, имеющий сужающуюся и расширяющуюся части (сопло Лаваля)

