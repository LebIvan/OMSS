\newpage
\section{Билет 10. Понятие температуры. Теплопроводность. Цикл Карно. Абсолютная температура. КПД цикла Карно. Второе начало термодинамики. Энтропия.}



\begin{center}
	\textit{\underline{Понятие температуры}}
\end{center}

А имеет большую температуру чем В,  если при контакте А и В тепловая энергия переходит от А к В. 
Температуру можно рассматривать как величину, пропорциональную средней энергии теплового движения молекул, приходящуюся на одну степень свободы.


\begin{center}
	\textit{\underline{Теплопроводность}}
\end{center}

Двухпараметрическая среда - это среда,  термодинамические функции которой зависят от двух параметров. Тогда удельная внутренняя энергия выражается через эти два параметра. Нам интересно, когда эти два параметра - это давление и плотность.

Если среда - это идеальная сжимаемая жидкость (газ), то работа внутренних поверхностных сил, отнесенная к единице массы, имеет вид
$$ \frac{1}{dm}dA_{face}^{(i)} = p\ d\frac{1}{\rho} $$
и уравнение притока тепла в предположении, что $q^{**} = 0$ (нет притока тепла извне) записывается следующим образом:
$$ dU + p\ d\frac{1}{\rho} = dq^{(e)} $$

В совершенном газе давление, плотность и температура связаны уравнением Клапейрона:
$$ p = \rho R T $$

R - газовая постоянная. Уравнение, связывающее физические характеристики среды - это \textbf{уравнение состояния}. Можно ввести универсальную газовую постоянную R и постоянную Больцмана так:
$$ R = \frac{R_0}{M} = \frac{k}{m} $$

М - средняя масса одной грамм-молекулы газа, k - полное число молекул в данном обьеме смеси,  m - средняя масса молекулы в граммах.

Совершенный газ можно определить как газ, в котором молекулы взаимодействуют только при столкновениях. Значит считать, что внутренняя энергия одноатомного совершенного газа представляет собой суммарную кинетическую энергию хаотического движения атомов. 

Для внутренней энергии U единицы массы можно написать
$$ U = \frac{1}{M}\sum\limits_{i=1}^{N}\frac{m_iv_i^2}{2} + const $$

Если считать, что все атомы газа одинаковые, то $M = Nm$ и 
$\displaystyle U = \frac{v_{average}^2}{2} + const $

Для совершенного газа удельную внутреннюю энергию U можно представить в виде
$$ U = c_VT + const \ \ \ \ (*)$$
$c_V$ - это размерный коэффициент пропорциональности между $\frac{1}{2}v_{average}^2$ и T
Задание внутренней энергии U в виде $(*)$ вместе с уравнением Клапейрона фиксирует модель сплошной среды, называемую совершенным газом. 

На основании уравнения притока тепла для совершенного идеального газа в случае процесса с постоянным удельным объемом $\left(\displaystyle d\frac{1}{\rho}=0\right)$ можно получить:
$$ (dq^{(e)})_{V = const} = dU = c_VdT $$
или
$$ \left( \frac{dq^{(e)}}{dT} \right)_{V=const} = c_V $$

Следовательно,  $c_V$ - это количество тепла,которое необходимо подвести к единице массы среды, чтобы при постоянном объеме поднять ее температуру на $1^oC$ (\textbf{теплоемкость при постоянном объеме}).

В случае процесса с постоянным давлением из уравнения притока тепла для идеального совершенного газа получим
$$ \left(d q^{(e)}\right)_{p=\text { const }}=d U+p d \frac{1}{\rho}=c_V d T+d \frac{p}{\rho}=\left(c_V+R\right) d T $$

Количество тепла, которое необходимо подвести к единице массы среды, чтобы при постоянном давлении поднять температуру на $1^oC$ - это \textbf{теплоемкость при постоянном давлении}, обозначается $C_p$:
$$ c_p = \left( \frac{dq^{(e)}}{dT} \right)_{p=const} $$

Следовательно, $\displaystyle c_p - c_V = R $


\begin{center}
	\textit{\underline{Цикл Карно}}
\end{center}

Процесс \textbf{адиабатический}, если нет притока внешнего тепла и теплообмена между соседними частицами ($\displaystyle dQ^{(e)} = 0$).
Процесс \textbf{изотермический}, если температуру всей системы можно считать постоянной ($\displaystyle \frac{dT}{dt} = 0$).

Рассмотрим равновесный обратимый замкнутый процесс (обратимый цикл Карно). В качестве среды возьмем совершенный газ или любую другую двухпараметрическую среду, определяемую параметрами р и $\frac{1}{\rho}$. 

Из произвольной точки $M (p_0, \frac{1}{\rho_0})$ пространства состояний газ по изотерме $\theta_1 = const$ расширяется до состояния $N$, затем расширяется адиабатически до состояния $K$ с температурой $\theta_2 < \theta_1$ и от $K$ сжимается изотермически до состояния $P$, из которого можно вернуться по адиабате в начальное состояние $M$.

\begin{figure}[H]
\includegraphics[width=0.5\textwidth]{carno.png}
\end{figure}


\begin{center}
	\textit{\underline{Второй закон термодинамики}}
\end{center}

\textbf{Формулировка 1:} Невозможно устройство, которое переводило бы тепло от тела с меньшей температурой к телу с большей температурой без изменений в других телах.

\textbf{Формулировка 2:} Нельзя построить вечный двигатель второго рода (машину,  которая,  работая в согласии с первым законом термодинамики по некоторому циклу,  периодически совершает работу только за счет охлаждения некоторого одного и того же источника тепла с фиксированной температурой (отбор тепла из резервуара с постоянной температурой)).

$$ A = Q_1 - Q_2 $$
$$ Q^{(e)} = \int\limits_{AB(l_1)}\left( dU + pd\frac{1}{\rho} \right) dm = \int\limits_{A}^{B}dU_m + A = U_{mB} - U_{mA} + A $$

По определению к.п.д.  цикла Карно - это отношение полученной в результате реализации цикла механической работы $A > 0$ к подведенному к системе за время цикла теплу $Q_1 > 0$.  Для к. п. д. цикла Карно верно
$$ \eta = \frac{A}{Q_1} = 1 - \frac{Q_2}{Q_1} \overset{1zt}{<} 1 $$

\begin{state}
	Для всякого обратимого цикла Карно величина $\eta$ зависит только от температур $\theta_1$ и $\theta_2$,  заданных на изотермах $MN$ и $KP$ и не зависит ни от свойств рабочего тела, участвующего в цикле Карно ни от способа организации цикла, определяемого, например, размерами рабочего тела и степенью расширения вдоль изотерм.
\end{state}

\begin{proof}
(От противного, противоречие со вторым законом термодинамики):

Пусть есть 2 цикла - обратимый ($\eta$) и необратимый ($\eta'$) c одинаковыми температурами $\theta_1 > \theta_2$. Пусть $\eta' > \eta$. Пусть машина с к.п.д. $\eta'$ работает в прямом направлении и проивзодит работу $A'$. Заставим обратимую машину работать в противоположном направлении.  Тогда для машины с к.п.д. $\eta'$ имеем $Q'_1 > 0$, $Q'_2 > 0$ и $A' = Q'_1 - Q'_2 > 0$.  А для машины с к.п.д. $\eta$ имеем $Q_1 > 0$, $Q_2 > 0$ и $A = Q_2 - Q_1 < 0$. Выберем обратимый цикл Карно так, чтобы $-A = A'$, т.е. $Q_1' - Q_2' = Q_1 - Q_2$ и соединим эти машины вместе. Получим машину, для которой $$ A_0 = A' + A = Q_1 + Q_2 - Q_1' - Q_2' $$

Эффект,производимый этой составной машиной, будет заключаться в перераспределении теплоты между нагревателем и холодильником. По построению (по выбору машин): $|A| = A' \ \ \Rightarrow \eta Q_1 = \eta'Q_1' $. Значит из предположения $\eta' > \eta$ следует $$ Q_1' < Q_1 $$
или $$ Q_1 - Q_1' = Q_2 - Q_2' > 0 $$
Здесь слева количество тепла,  передаваемое в резервуар с более высокой температурой, а справа - равна общему количеству тепла, забираемому из резервуара с температурой $\theta_2$

Таким образом, составная машина без затраты внешней энергии будет переводить тепло от холодного резервуара к горячему, что невозможно согласно второму закону термодинамики.

При доказательстве мы не пользовались ни свойствами рабочего тела ни частными свойствами цикла, следовательно, к.п.д.  обратимого цикла Карно не зависит от свойств рабочего вещества и от степени расширения,  а зависит только от $\theta_1$ и $\theta_2$ и является универсальной функцией $\eta = \eta(\theta_1, \theta_2)$

\end{proof} 

Найдем эту универсальную функцию.  По определению к.п.д. цикла Карно имеем
$$ \eta(\theta_1, \theta_2) = \frac{A}{Q_1} = 1 - \frac{Q_2}{Q_1} $$
Введем функцию $f(\theta_1, \theta_2) = 1 - \eta(\theta_1, \theta_2) = \frac{Q_2}{Q_1}$

Рассмотрим три тела большой теплоемкости с температурами $\theta_1, \theta_2, \theta_3$ и три обратимых цикла Карно, в которых эти тела служат нагревателями или холодильниками.
$$ f(\theta_1, \theta_2) = \frac{Q_2}{Q_1} = \frac{Q_2}{Q_3}\frac{Q_3}{Q_1} = f(\theta_3, \theta_2)f(\theta_1, \theta_3) $$

В случае $\theta_1 = \theta_2$:
$$ 1 = f(\theta_3, \theta_1)f(\theta_1, \theta_3) $$
То есть при перестановке аргументов функция обращается. Следовательно, 
$$ \frac{Q_2}{Q_1} = f(\theta_1, \theta_2) = \frac{f(\theta_3, \theta_2)}{f(\theta_3, \theta_1)} \ \ \ \ (**)$$
Отсюда следует, что $\frac{Q_2}{Q_1}$ не зависит от $\theta_3$.  Решение функционального уравнения $(**)$ имеет вид
$$ f(\theta_1, \theta_2) = \frac{\omega(\theta_2)}{\omega(\theta_1)} $$
Следовательно $$ \frac{Q_2}{Q_1} = \frac{\omega(\theta_2)}{\omega(\theta_1)} $$

\begin{defn}
	Абсолютная температура $T$ - значение функции $\omega(\theta)$
\end{defn}

Тогда $$ \frac{Q_2}{Q_1} = \frac{T_2}{T_1} $$

\textbf{Формулировка 3 (количественная для обратимого цикла Карно):} 
$$ \frac{Q_1^{(e)}}{T_1} + \frac{Q_2^{(e)}}{T_2} = 0 $$

\begin{center}
	\textit{\underline{Энтропия}}
\end{center}

Фиксируя точку начального состояния системы А для любого состояния В двухпараметрической среды, в которое можно перейти из состояния А обратимыми путями, можно ввести функцию параметров состояния - координат точки В:
$$ S(B) = S\left(p, \frac{1}{\rho}\right) = \int\limits_{A}^{B}\frac{dQ^{(e)}}{T} + S(A) $$

\begin{defn}
	 Энтропия - это функция $S(B)$
\end{defn}

Из определения следует, что $\displaystyle dS = \frac{dQ^{(e)}}{T} $

Из уравнения притока тепла:
$\displaystyle dS = \frac{dU_m + dA^{(i)}}{T} $

Или в рассчете на единицу массы:
$\displaystyle ds = \frac{dq^{(e)}}{T} = \frac{dU + pd\frac{1}{\rho}}{T} $  \\
\textbf{Случай совершенного газа:}

Для совершенного газа с постоянными теплоемкостями ($p = \rho R T, \ U = c_V T$) будем иметь
$$ ds = \frac{c_VdT}{T} + \frac{Rd\frac{1}{\rho}}{\frac{1}{\rho}} = dln\left[T^{c_V}\left( \frac{1}{\rho} \right)^{R}\right] $$




