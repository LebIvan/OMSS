\newpage
\newpage
\section {Билет 9. Изменение энергии в конечном объеме сплошной среды (первое начало термодинамики). Работа и внутренняя энергия. Уравнение притока тепла.}
\begin{center}
	\textit{\underline{Кратко}}
\end{center}
Основные обозначения:
\begin{itemize}
    \item $dE$ -- изменение кинетической энергии рассматриваемого тела;
    \item $U$ -- плотность внутренней энергии (скалярная функция параметров состояния);
    \item $dU_m$ -- изменение внутренней энергии рассматриваемого тела;
    \item $dA^{(e)}$ -- элементарная работа внешних сил;
    \item $dA^{(i)}$ -- элементарная работа внутренних сил;
    \item $dQ^{*}$ -- элементарный приток энергии к телу извне;
    \item $dQ^{(e)}$ -- элементарный приток тепла к телу извне;
    \item $dQ^{**}$ -- элементарный приток нетепловых видов энергии к телу.
\end{itemize}
Закон сохранения энергии:
$$
dE + dU_m = dA^{(e)} + dQ^{(e)} + dQ^{**}
$$
Полная энергия частицы:
$$
\mathcal{E} = (E + U)\rho d\tau
$$
Уравнение притока тепла:
$$
dU_m = -dA^{(i)} + dQ^{(e)} + dQ^{**}
$$

\begin{center}
	\textit{\underline{Изменение энергии в конечном объеме сплошной среды (первое начало термодинамики)}}
\end{center}
Допустим, имеется процесс, протекающий в пространстве состояний от точки $A = (\mu^i_0)$ до точки $B = \mu^i$ по некоторой кривой $L$. Полный приток энергии равен:
$$
A^{(e)} + Q^{*} = \int_{AB(L)}P_id\mu^i + \int_{AB(L)}Q_id\mu^i
$$
Оказывается, что приток энергии не зависит от $L$ из-за фундаментального закона природы, в частности постулирующего, что:
$$
\oint_{C}(P_i + Q_i)d\mu^i = 0
$$
То есть, что полный приток энергии, поступающий извне к системе, совершающей любой осуществимый цикл, равен нулю.
\begin{center}
	\textit{\underline{Работа и внутренняя энергия}}
\end{center}
Полная энергия частицы:
$$
\mathcal{E} = (E + U)\rho d\tau
$$
Если внутренняя энергия аддитивна, то полная энергия произвольного конечного объема V равна:
$$
\mathcal{E} = \int_{V}\rho(\frac{v^2}{2} + U)d\tau
$$
\begin{center}
	\textit{\underline{Уравнение притока тепла}}
\end{center}
Вычтя из закона сохранения энергии равенство, выражающее теорему живых сил, получим \textbf{уравнение притока тепла:}
$$
dU_m = -dA^{(i)} + dQ^{(e)} + dQ^{**}
$$
или
$$
dU_m = -dA^{(i)} + dQ^{*}
$$
Полагая для квазистатических процессов $dA^{(e)} = -dA^{(i)}$, получим уравнение:
$$
dU_m = dA^{(e)} + dQ^{*}
$$
